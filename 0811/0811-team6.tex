\documentclass[a4paper, 11pt, nofonts, nocap, fancyhdr]{ctexart}

\usepackage[margin=60pt]{geometry}

\setCJKmainfont[BoldFont={方正黑体_GBK}, ItalicFont={方正楷体_GBK}]{方正书宋_GBK}
\setCJKsansfont{方正黑体_GBK}
\setCJKmonofont{方正仿宋_GBK}

\CTEXoptions[today=small]

\pagestyle{plain}

\renewcommand{\thesubsubsection}{Problem \Alph{subsubsection}.}
\newcommand{\problem}[1]{\subsubsection{#1}}

\title{Fudan ACM-ICPC Summer Training Camp 2015}
\author{Team 6 汤定一/马天翼/金杰}
\date{2015年8月11日}

\begin{document}

\maketitle

\section{概况}

本场训练,我们队伍在比赛中完成了5道题目,比赛后完成了6道题目,共完成11道题目。

\section{训练过程}

mty写05,忘删调试信息挂一次,string访问越界挂两次,jj重写05,通过。tdy写07,一次通过。tdy写11,一次通过。mty写03,在算法没有被证明的情况下被jj怂恿去写,wa。tdy提出了正确的解法,mty修改后通过。此时162min。jj写10,没过样例,下来读代码。mty与tdy在jj写10的时候想题,想出了02和04的解法,写02。jj修改10后1次通过。mty与tdy共同写02,将04解法告诉jj。02两次提交后wa,读代码。jj写04,此时剩40min,觉得写不完,弃疗。mty与tdy多次修改02代码,在赛后15min通过。

\section{解题报告}

\problem{Game On the Tree}

\begin{description}
\item[负责] 金杰
\item[情况] 比赛后通过
\item[题意]
在树上找一条路径,用这条路径上的值当P进制数的每一位,求最大值,输出方案。
\item[题解]
即,从树上找一条从非0点开始的最长路径,若有多条选择字典序最大的。\\
树的中心:树的直径的中间点。最长路径一定经过树的中心,如果不经过,把路径后半段扳过来强行经过树心一定更长。\\
首先找树心,随便找个点dfs一遍找到最远的,从最远的开始dfs一遍找到最远的。于是得到直径,取中间点(1或2个)就是树心。\\
对每个树心,先dfs一遍找到最远的非0起点们。然后一起bfs回溯得字典序最大的。删掉起点所在子树,再找最远的字典序最大的,就是终点。然后统计答案即可。

\end{description}

\problem{Tree Maker}

\begin{description}
\item[负责] 汤定一
\item[情况] 赛后通过
\item[题意]
给定一棵树,它初始时只有根节点,给定5种操作,1.跳到根节点2.跳到左儿子3.跳到右儿子4.在左儿子新建x个结点的子树5.在右儿子新建x个结点的子树。统计不同的树的形态的个数。
\item[题解]
树形动规。我们知道它走过的树的结点。每个4、5操作可以在树上的一个范围新建一些结点,对这些范围分别进行动规再乘起来即可。f[u][i+j]=f[l][i]*f[r][j],u是父节点,f[l][i]是左子树新建i个结点的方案数,f[r][i]是右子树新建j个结点的方案。
\end{description}

\problem{Hotaru's problem}

\begin{description}
\item[负责] 马天翼、汤定一、金杰
\item[情况] 比赛中通过 - 202min(5Y)
\item[题意]
给出n个数,求最长的连续的一列数,使把它分成三部分,前两部分对称,第一部分和第三部分相同。
\item[题解]
先用manacher求出以每个空为中心的最大回文串的长度,然后从最后开始扫,将他们加进set和vector,一旦到了一个位置使后面的某个点无法覆盖到这个点,则将其删除。求答案即可。
\end{description}

\problem{Segment Game}

\begin{description}
\item[负责] 汤定一
\item[情况] 赛后通过
\item[题意]
在一维上放线段。有两种操作:1.在[p,p+i]放一条长为i的线段,i为第i次1操作。2.删除第i次加入的线段。每次1操作都输出[p,p+i]内完整线段的个数
\item[题解]
因为每次加入操作是从短到长加入的。每次统计只需统计[l,r]内右端点的个数减去跨越了l的右端点个数,即减去[0,l-1]左端点个数与[0,l-1]右端点个数的差值。因此用两个树状数组即可。
\end{description}

\problem{The shortest problem}

\begin{description}
\item[负责] 金杰、马天翼
\item[情况] 比赛中通过 - 51min(4Y)
\item[题意]
给一个数,计算各位数之和,并加到这个数最后,重复以上操作t次,问最后所得数是否能被11整除。
\item[题解]
能被11整除的条件是奇数位之和与偶数位之和之差能被11整除,因此只要维护奇数位之和和偶数位之和即可。
\end{description}

\problem{Tetris}

\begin{description}
\item[负责] 马天翼
\item[情况] 比赛后通过
\item[题意]
给定俄罗斯方块的操作和种类,问消了几次。
\item[题解]
暴力模拟。
\end{description}

\problem{Gray code}

\begin{description}
\item[负责] 汤定一
\item[情况] 比赛中通过 - 61min(1Y)
\item[题意]
给定二进制码,每位为'0','1','?',问它所对应的格雷码分数最高是多少,若格雷码第i位为1即可获得a[i]分。
\item[题解]
格雷码与二进制码的转换关系为x xor x>>1,第i位格雷码只与第i位与第i+1位二进制码有关。动规f[i][j],第i位格雷码为j的最大分数。
\end{description}

\problem{Convex Polygon}

\begin{description}
\item[负责] 汤定一
\item[情况] 赛后通过
\item[题意]
给定n,m,k,求从正n边形中选m个顶点,形成的多边形中正好有k个锐角的方案数。
\item[题解]
分5种情况讨论,详情请看代码。
\end{description}

\problem{Root}

\begin{description}
\item[负责] 汤定一
\item[情况] 赛后通过
\item[题意]
给定sum和m,m个询问,每次询问给定x、y,求使得$x^k=y$在mod p下最小的k。p是sum的质因子,只要满足一个p即可。
\item[题解]
$x^k$ mod p=y,可以转化为klogd(x)mod(p-1)=logd(y),可以用欧几里得扩展求得最小的k,其中d为p的原根。logd(x)的求解可以转化为$d^{kx}$ mod p=x,可以用baby step giant step。
\end{description}

\problem{Leader in Tree Land}

\begin{description}
\item[负责] 金杰
\item[情况] 比赛中通过 - 230min(1Y)
\item[题意]
给一棵树上每个点标上1~n的值,每一棵子树的leader是该子树数最大的。问共有k个leader的方案数。
\item[题解]
树DP。\\
f[i][j]表示访问到i点之前的所有先序遍历比i早访问过的点中有j个leader的方案数。每次向儿子DP时,乘上可取标号数的组合数,回溯时传递上来,反正父亲已经处理完了没有用了,方便下个点计算。\\
DP方程为:\\
f[v][j]=f[u][j-1]*C[now][sz[v]]\\
f[v][j]+=f[u][j]*C[now][sz[v]]*(sz[v]-1)\\
回溯时f[u][j]=f[v][j]\\
now为可用标号数

\end{description}

\problem{Mahjong tree}

\begin{description}
\item[负责] 汤定一
\item[情况] 比赛中通过 - 118min(1Y)
\item[题意]
给定一棵树,给每个结点分配1-n个苹果,要满足两个要求:1.以一个结点为根的子树苹果个数连续2.一个结点的儿子苹果个数连续。求可行的方案数。
\item[题解]
树形动规。一个结点若有x个单个的儿子节点,这个点的答案乘上x!。父节点乘上所有儿子的方案数,若一个结点有一个以上的非单个结点,该节点答案乘以2。若n>1,根节点答案乘以2。
\end{description}


\section{总结}

递归的时候注意全局变量是否更改。

\end{document}

