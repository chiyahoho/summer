\documentclass[a4paper, 11pt, nofonts, nocap, fancyhdr]{ctexart}

\usepackage[margin=60pt]{geometry}

\setCJKmainfont[BoldFont={方正黑体_GBK}, ItalicFont={方正楷体_GBK}]{方正书宋_GBK}
\setCJKsansfont{方正黑体_GBK}
\setCJKmonofont{方正仿宋_GBK}

\CTEXoptions[today=small]

\pagestyle{plain}

\renewcommand{\thesubsubsection}{Problem \Alph{subsubsection}.}
\newcommand{\problem}[1]{\subsubsection{#1}}

\title{Fudan ACM-ICPC Summer Training Camp 2015}
\author{Team 6 汤定一/马天翼/金杰}
\date{2015年8月14日}

\begin{document}

\maketitle

\section{概况}

本场训练,我们队伍在比赛中完成了7道题目,比赛后完成了2道题目,共完成9道题目。

\section{训练过程}

开局mty写A,一次通过。tdy和jj读题。然后jj写B,一次通过。tdy把C题意告诉jj,一起想了一会儿,jj决定读一遍C,发现只有往前的限制,写C。tdy读题,想出F。C没过样例,测样例发现算法有问题,tdy去写F,至半换jj写C,一次通过,然后F一次通过。mty在写完A后推了D,在F通过后开始写D,然后wa了。jj和tdy读题想题,发现E是模拟题,准备等mty写完D交给他写E。然后jj想出了G,在D提交后开始写G,RE。由于D仍然未过,tdy读过后面的题后觉得都不太可做,开始写E。期间my和jj不停修改提交。jj实在无法发现错误,于是非常弃疗的改了没影响的东西再交,白白多RE了两次,最后发现是清空问题,AC。mty一直在推导物理计算过程是否错误,修改了许多地方仍然wa。在G通过后把代码交给jj看,发现是初始最大值不够大的问题,修改后通过。然后tdy继续写E,一次通过。此时还有1小时,三个人一起想I。jj推出了一个很奇怪的DP方程,因为没有时间没有证明就上去写了,mty和tdy继续想,认为jj的做法是有后效性的,但因为没有明确清楚jj的做法而且时间很紧就没有说。jj写到一半觉得肯定错了,没过样例后又弃疗式更改一些地方还是错误,事实证明只要倒过来做DP就可以了,应该在写之前跟队友讲清楚做法。

\section{解题报告}

\problem{Bit String Reordering}

\begin{description}
\item[负责] 马天翼
\item[情况] 比赛中通过 - 25min(1Y)
\item[题意]
给一个数列,表示一串由01组成的数列的连续的0和1的个数,再给出一个初始序列,只能交换相邻两个数,问最少交换几次得到目标数列。
\item[题解]
有两个可能的数列,一位一位交换到位即可。
\end{description}

\problem{Miscalculation}

\begin{description}
\item[负责] 金杰
\item[情况] 比赛中通过 - 35min(1Y)
\item[题意]
给个只有+*的表达式,问是从左到右计算还是先*后+。
\item[题解]
后缀表达式。
\end{description}

\problem{Shopping}

\begin{description}
\item[负责] 金杰
\item[情况] 比赛中通过 - 63min(1Y)
\item[题意]
从0走到n+1,中间有n家商店,有一些<u,v>,u<v,去u之前必须去过v,问多久能逛完到n+1。
\item[题解]
如果有两条关系有重叠部分,则并成一条,那么每条都无重叠了,先走到最大要去的,然后到最小之前不能去的,再往前走即可。
\end{description}

\problem{Space Golf}

\begin{description}
\item[负责] 马天翼
\item[情况] 比赛中通过 - 180min(7A)
\item[题意]
给一个初始点和目标点,发射一枚子弹,做抛体运动,与地接触后发生完全弹性碰撞,给出最大反弹次数,中间有一些高度不同的障碍物,问初速度最小为多少,使子弹能到达目标点。
\item[题解]
推出公式二分即可。
\end{description}

\problem{Automotive Navigation}

\begin{description}
\item[负责] 汤定一
\item[情况] 比赛中通过 - 239min(1Y)
\item[题意]
给定n条路组成的城市,m条记录。每条记录记录了走过的里程数以及当前的方向。问目前可能在哪个坐标上。
\item[题解]
模拟题。50*50*4,城市大小50*50,4个方向。
\end{description}

\problem{There is No Alternative}

\begin{description}
\item[负责] 汤定一
\item[情况] 比赛中通过 - 80min(1Y)
\item[题意]
给定n个点,m条边,问哪些边是最小生成树的必须边。
\item[题解]
Kruscal,若u,v在不同集合,合并两个集合,否则把u到v的路径上所有与当前边边权相等的边标记为非必须边即可。
\end{description}

\problem{Flipping Parentheses}

\begin{description}
\item[负责] 金杰
\item[情况] 比赛中通过 - 177min(4Y)
\item[题意]
给一个合法括号序列。每次改变一个括号,问改变最左边的能使序列重回合法的括号位置。
\item[题解]
(为1,)为-1,前缀和,合法条件为每个点都>0。\\
若将一个1改-1,则把最左边的-1改1即可,用set维护。\\
若将一个-1改1,则把最右边的<2的点改成-1即可,要用线段树维护区间最小值。
\end{description}

\problem{Cornering at Poles}

\begin{description}
\item[情况] 尚未通过
\end{description}

\problem{Sweet War}

\begin{description}
\item[负责] 金杰
\item[情况] 比赛后通过
\item[题意]
n(<150)颗巧克力们在栈里,每次只能吃最上面的那颗。每颗巧克力有能量值(<1e9)和分数(<150)。A和B有初始能量值,轮流操作,可以选择吃掉最上面这个,增加该巧克力的能量和分数,可以选择不吃,则消耗1点能量值。问最后两人的分数。
\item[题解]
倒着做,f[i][j]表示当前第i颗,得到j分数,先手,至少需要多少体力值。转移过程中取min。最后看f[1][j]最大的j使得小于A-B。
\end{description}

\problem{Exhibition}

\begin{description}
\item[负责] 马天翼
\item[情况] 比赛后通过
\item[题意]
有n个产品,每个产品有三个属性x,y,z,从中选k个使($x_{i1}+x_{i2}+...+x_{ik}$)*($y_{i1}+y_{i2}+...+y_{ik}$)*($z_{i1}+z_{i2}+...+z_{ik}$),现在可以修改第一个产品的每一个属性,但需要付出百分比的某个数的代价,问最小花费多少代价,使第一个产品有可能被选中。
\item[题解]
将公式*3,可以的得到一个Σ(Px+Qy+Rz)的公式,我们发现这是两个向量的点乘,即一个向量在另一个向量上的投影。那么问题转化成了,求在(P,Q,R)向量投影上的前k小的向量的公式和。然后我们枚举空间中的n三次方个空间进行判断,再进行12次讨论即可(太烦了,实在讲不出来)。
\end{description}

\problem{L infinity Jumps}

\begin{description}
\item[情况] 尚未通过
\end{description}

\section{总结}

在时间很紧的情况下,是否要跟队友充分交流似乎是个没有明确策略的抉择。\\
永远相信题目和评测机的正确,不要怀疑逻辑正确的写法能不能这么写,不要弃疗式提交。

\end{document}


