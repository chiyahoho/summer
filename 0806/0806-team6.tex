\documentclass[a4paper, 11pt, nofonts, nocap, fancyhdr]{ctexart}

\usepackage[margin=60pt]{geometry}

\setCJKmainfont[BoldFont={方正黑体_GBK}, ItalicFont={方正楷体_GBK}]{方正书宋_GBK}
\setCJKsansfont{方正黑体_GBK}
\setCJKmonofont{方正仿宋_GBK}

\CTEXoptions[today=small]

\pagestyle{plain}

\renewcommand{\thesubsubsection}{Problem \Alph{subsubsection}.}
\newcommand{\problem}[1]{\subsubsection{#1}}

\title{Fudan ACM-ICPC Summer Training Camp 2015}
\author{Team 6 汤定一/马天翼/金杰}
\date{2015年8月6日}

\begin{document}

\maketitle

\section{概况}

本场训练,我们队伍在比赛中完成了4道题目,比赛后完成了4道题目,共完成8道题目。

\section{训练过程}

mty发现了K题,与jj讨论后立刻写K。7min提交,OLE,找错改地方,仍然OLE。tdy重写K,21min通过(后K题rejudge之前的代码都A了)。jj读A,写,WA。tdy写F,TLE,此时74min。改A提交WA,弃题。mty想出C的结论,写C,一次通过。jj写H,一次通过,此时124min。tdy找到瓶颈,位运算优化,仍然TLE。jj想出F的一个优化,非常麻烦,tdy修改后调试,修过bug后仍然TLE。mty读A题代码,找错,未果,重写A,WA,此时233min。jj用线段树写I,WA,此时278min,还剩20分钟,决定放弃I(事后知道线段树会T,要用堆)。mty发现A的错误,tdy立刻想到了用拓扑排序修A,jj去写,AC。tdy在最后一小时写了B,写完发现做法错误。

\section{解题报告}

\problem{Average}

\begin{description}
\item[负责] 金杰
\item[情况] 比赛中通过 - 296min(6Y)
\item[题意]
n个人围一圈,任意两个相邻的人可以传递一次一颗糖果。问能否使所有人糖果数相同,输出方案。
\item[题解]
减去平均数,枚举第n个人给第1个人多少糖果,从1开始扫一遍,维护给下一个人多少糖果。因为有人可能没有糖果,要在拿到糖果后再传出去,所以输方案时要先建图拓扑排序。
\end{description}

\problem{Bipartite Graph}

\begin{description}
\item[负责] 汤定一
\item[情况] 赛后通过
\item[题意]
给定n个点,m条边,问在删哪个点的情况下此图为二分图,求所有这样的点。
\item[题解]
若一个点被删,那么与它相连的边都失效。一条边(u,v)(u<v)在1~n上的有效域为(1~u-1),(u+1~v-1),(v+1~n)我们在1~n上做文章。用类似线段树的思想,若一条边的一个作用域为(s,t)对当前的线段树区间(l,r)若s==l且t==r则加入这个区间,令mid=(l+r)/2,若t<=mid则加入左区间,若s>mid则加入右区间,若(s,t)跨过mid,则(s,mid)加入左区间(mid+1,t)加入右区间。dfs这棵线段树,当递归到每一个区间时加入线段去判断是否仍为二分图,递归时则恢复,可以用栈维护。判断是否为二分图可用并查集维护,不用路径压缩,启发式合并。总时间复杂度$mlogn^2$
\end{description}

\problem{Cake}

\begin{description}
\item[负责] 马天翼、金杰
\item[情况] 比赛中(后?)通过 - 92min(1Y)
\item[题意]
给1~n的数,问能不能分到m块里使总和相等。
\item[题解]
比赛中是mty做的,把n个数字倒着放到m块里能放就放,就AC了。赛后改了数据。jj重写,改成每次挑剩余空间最多的块放,如果放哪块能刚好填满就放那块,就AC了。数据太水了。。
\end{description}

\problem{Deal}

\begin{description}
\item[情况] 尚未通过
\end{description}

\problem{Easy Sequence}

\begin{description}
\item[负责] 汤定一、金杰
\item[情况] 比赛后通过
\item[题意]
给一个括号序列,$ans_i$=包含第i个字符的合法子串数。问sum of i*$ans_i$。
\item[题解]
$match_i$为i的匹配括号。$a_i$为以i开始的方案数。$b_i$为以i结尾的方案数。a[i]=a[match[i]+1]+1, b[i]=b[match[i]-1]+1;\\
$up_i$为最小的包含了i和match[i]的合法序列。ans[i]=ans[match[i]]=ans[up[i]]+a[i]*b[match[i]];
\end{description}

\problem{First One}

\begin{description}
\item[负责] 汤定一
\item[情况] 赛后通过
\item[题意]
给定序列A,for i from 1 to n for j from i to n (floor(log2(S(i,j))+1)*(i+j)
\item[题解]
注意到从i开始的字段和log2(S(i,j))最多有35个值,于是对于每个i,把i-1的区间往后推即可,因为是单调的。时间复杂度(nlogn)
\end{description}

\problem{Group}

\begin{description}
\item[情况] 尚未通过
\end{description}

\problem{Hiking}

\begin{description}
\item[负责] 金杰
\item[情况] 比赛中通过 - 124min(1Y)
\item[题意]
你尝试邀请n个人参加聚会,每个人答应参加的条件是此时去的人数为Li~Ri之间,给一个邀请顺序使参会者最多。
\item[题解]
以L为关键字排序,一旦L小于当前人数就将此人加入堆,堆按R排序小根堆,每次取一个R最小的出来看看会不会参加。
\end{description}

\problem{In Touch}

\begin{description}
\item[负责] 金杰
\item[情况] 比赛后通过
\item[题意]
n个位置一排,位置i可以用Ci花费传送到距离为Li~Ri的地方,问从1到每个位置的最小花费。
\item[题解]
用类似dijkstra的方法做,每次更新就往堆里插一个node表示更新的一段,记录左端点右端点和更新的距离。然后每次从堆中取出一段来,从l到r扫一遍去更新。为了将扫的效率降至log,在外面开一个1~n的set表示此点未确定,每次扫完一个从set里删掉,扫l~r的时候就能用lowerbound了。
\end{description}

\problem{Just A String}

\begin{description}
\item[情况] 比赛后通过
\end{description}

\problem{Key Set}

\begin{description}
\item[负责]汤定一、马天翼
\item[情况] 比赛中通过 - 7min(1Y)
\item[题意]
给定集合{1,2,……,n},求n的子集中和为奇数的个数
\item[题解]
$2^{n-1}-1$
\end{description}

\section{总结}

不要一头栽进一道题里。

\end{document}