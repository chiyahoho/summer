\documentclass[a4paper, 11pt, nofonts, nocap, fancyhdr]{ctexart}

\usepackage[margin=60pt]{geometry}

\setCJKmainfont[BoldFont={方正黑体_GBK}, ItalicFont={方正楷体_GBK}]{方正书宋_GBK}
\setCJKsansfont{方正黑体_GBK}
\setCJKmonofont{方正仿宋_GBK}

\CTEXoptions[today=small]

\pagestyle{plain}

\renewcommand{\thesubsubsection}{Problem \Alph{subsubsection}.}
\newcommand{\problem}[1]{\subsubsection{#1}}

\title{Fudan ACM-ICPC Summer Training Camp 2015}
\author{Team 6 汤定一/马天翼/金杰}
\date{2015年8月18日}

\begin{document}

\maketitle

\section{概况}

本场训练,我们队伍在比赛中完成了3道题目,比赛后完成了N道题目,共完成N道题目。

\section{训练过程}

开局mty写03,wa了以后发现没那么简单。jj写05,没过样例,喊tdy马上上来重写。jj去读题,发现04是水题,准备写。tdy因为公式推错wa2发,ij打反wa1发,74min通过。jj觉得好像没写错,把04代码交给mty看,mty也觉得题意和代码都没错,于是放弃。tdy写01,jj和mty搞07。tdy发现公式错了,mty去写07。tdy回来写01,仍然未过,发现是题意读错,于是再换07。期间jj在读题和想03和08。jj莫名想到04中的一个特例,于是修改后通过,此时还有2小时。mty和jj一起看07代码找错,终于改对之后提交,TLE。然后jj想到07一个简单的结论,于是省了一个搜索,但tdy在机子上写01。01在第四个小时通过,还剩1小时。mty修改07之后,wa了,修改了几次仍然wa,因为代码是之前的改的,所以很长而且逻辑混乱,jj决定重写07,此时还有40min。tdy想出了03,在强调肯定没有错而且写的很快之后,jj将机子让给tdy。03没有过样例,于是jj继续重写07,此时还有20min。07没过样例,换03,此时还有8min。07读代码找到错误,在还有4min提交了一次,wa了。剩下4min继续尝试03,至结束未通过。07在赛后调试10min后通过。

\section{解题报告}

\problem{我是A题的标题}

\begin{description}
\item[负责] 负责一、负责二
\item[情况] 比赛中通过 - 233min(1Y)
\item[题意]
我是题意。\\这是换行
回车不换行。
\item[题解]
我是题解。
\end{description}

\problem{我是B题的标题}

\begin{description}
\item[负责] 负责一、负责二
\item[情况] 比赛中通过 - 233min(1Y)
\item[题意]
我是题意。
\item[题解]
我是题解。
\end{description}

\problem{我是C题的标题}

\begin{description}
\item[负责] 负责一、负责二
\item[情况] 比赛后通过
\item[题意]
我是题意。
\item[题解]
我是题解。
\end{description}

\problem{Too Simple}

\begin{description}
\item[负责] 金杰
\item[情况] 比赛中通过 - 175min(4Y)
\item[题意]
有m个从集合1-n到集合1-n的映射,有一些映射不知道,表示成-1,问最后有多少方案能使所有的i经过所有映射后等于i。
\item[题解]
如果有多个-1,则其他-1随便排,让最后一个-1排成满足答案的即可。如果没有-1,则直接判断是否满足答案。注意,即使有-1,如果某个映射使集合size变小了也是永远不能满足答案的。
\end{description}

\problem{我是E题的标题}

\begin{description}
\item[负责] 负责一、负责二
\item[情况] 比赛中通过 - 233min(1Y)
\item[题意]
我是题意。
\item[题解]
我是题解。
\end{description}

\problem{我是F题的标题}

\begin{description}
\item[负责] 负责一、负责二
\item[情况] 比赛中通过 - 233min(1Y)
\item[题意]
我是题意。
\item[题解]
我是题解。
\end{description}

\problem{Travelling Salesman Problem}

\begin{description}
\item[负责] 金杰、马天翼
\item[情况] 比赛后通过
\item[题意]
n*m的棋盘上每个点都有一个非负的数,问从左上角走到右下角最大能收集到的数字和。
\item[题解]
如果n或m是奇数,则走蛇形即可。否则至少有1个点会到不了。如果i+j为奇数,发现一定能构造只牺牲这个点的方案。如果i+j为偶数,发现至少牺牲另一个i+j为奇数的点,所以绝对会经过所有偶数点。于是找奇数点里值最小的即可。
\end{description}

\problem{我是H题的标题}

\begin{description}
\item[负责] 负责一、负责二
\item[情况] 比赛中通过 - 233min(1Y)
\item[题意]
我是题意。
\item[题解]
我是题解。
\end{description}

\problem{我是I题的标题}

\begin{description}
\item[负责] 负责一、负责二
\item[情况] 比赛中通过 - 233min(1Y)
\item[题意]
我是题意。
\item[题解]
我是题解。
\end{description}

\problem{我是J题的标题}

\begin{description}
\item[负责] 负责一、负责二
\item[情况] 比赛中通过 - 233min(1Y)
\item[题意]
我是题意。
\item[题解]
我是题解。
\end{description}

\problem{我是K题的标题}

\begin{description}
\item[负责] 负责一、负责二
\item[情况] 比赛中通过 - 233min(1Y)
\item[题意]
我是题意。
\item[题解]
我是题解。
\end{description}

\problem{我是L题的标题}

\begin{description}
\item[负责] 负责一、负责二
\item[情况] 比赛中通过 - 233min(1Y)
\item[题意]
我是题意。
\item[题解]
我是题解。
\end{description}

\section{总结}

我是总结。

\end{document}


