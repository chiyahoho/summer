\documentclass[a4paper, 11pt, nofonts, nocap, fancyhdr]{ctexart}

\usepackage[margin=60pt]{geometry}

\setCJKmainfont[BoldFont={方正黑体_GBK}, ItalicFont={方正楷体_GBK}]{方正书宋_GBK}
\setCJKsansfont{方正黑体_GBK}
\setCJKmonofont{方正仿宋_GBK}

\CTEXoptions[today=small]

\pagestyle{plain}

\renewcommand{\thesubsubsection}{Problem \Alph{subsubsection}.}
\newcommand{\problem}[1]{\subsubsection{#1}}

\title{Fudan ACM-ICPC Summer Training Camp 2015}
\author{Team 6 汤定一/马天翼/金杰}
\date{2015年8月3日}

\begin{document}

\maketitle

\section{概况}

开局mty写J,一次通过。jj写H,wa了。tdy看了H的题目,坚持认为jj读错题,而jj认为没有理解错,在jj将信将疑下,写了,通过。mty和tdy讨论后,mty去写E。期间jj和tdy想题,攒下了许多题。90min,E题wa了,tdy马上上去写F。没过样例,调试10min,换jj去写G。tdy在机下发现错误,修改后1次通过,然后jj的G1次通过,此时过去两个半小时。mty一直在做E。tdy和jj都认为在B在残量网络上继续跑就可以,于是tdy去写B,然而TLE。tdy想出了I的做法,将解法告诉jj,jj写I,一次通过,此时3小时22分。因为E消耗了太多时间,mty放弃E,去找B的模板。tdy在mty的帮助下来重写E,通过。mty找到了B的模板觉得应该是的,交给jj,去帮忙搞E了。jj抄完模板,通过。

\section{训练过程}

我是过程。

\section{解题报告}

\problem{我是A题的标题}

\begin{description}
\item[负责] 负责一、负责二
\item[情况] 比赛中通过 - 233min(2Y)
\item[题意]
我是题意。\\这是换行
回车不换行。
\item[题解]
我是题解。
\end{description}

\problem{Nubulsa Expo}

\begin{description}
\item[负责] 金杰、马天翼、汤定一
\item[情况] 比赛中通过 - 280min(2Y)
\item[题意]
给无向图,从中找一个点当汇点,问最小的从1到汇点的最大流。
\item[题解]
最大流即最小割,1一定在割的一边,所以即是求无向图全图最小割。模板题。
\end{description}

\problem{Shade of Hallelujah Mountain}

\begin{description}
\item[负责] 金杰
\item[情况] 比赛后通过
\item[题意]
我是题意。
\item[题解]
我是题解。
\end{description}

\problem{我是D题的标题}

\begin{description}
\item[负责] 负责一、负责二
\item[情况] 尚未通过
\item[题意]
我是题意。
\item[题解]
我是题解。
\end{description}

\problem{我是E题的标题}

\begin{description}
\item[负责] 负责一、负责二
\item[情况] 比赛中通过 - 233min(1Y)
\item[题意]
我是题意。
\item[题解]
我是题解。
\end{description}

\problem{我是F题的标题}

\begin{description}
\item[负责] 负责一、负责二
\item[情况] 比赛中通过 - 233min(1Y)
\item[题意]
我是题意。
\item[题解]
我是题解。
\end{description}

\problem{Farm Game}

\begin{description}
\item[负责] 金杰
\item[情况] 比赛中通过 - 144min(1Y)
\item[题意]
给n种作物的价格和数量。再给很多1个a能换k个b的转换关系。问最多能卖多少钱。
\item[题解]
因为保证无环,是拓扑图,倒着推,把价格更新过去就好了。最后统计一遍。
\end{description}

\problem{Selecting courses}

\begin{description}
\item[负责] 金杰
\item[情况] 比赛中通过 - 88min(2Y)
\item[题意]
给n门课的开放时间,学生每选一节课,下一次选课时间是5分钟后,其他时间无效,问最多多少门课。
\item[题解]
枚举第一门课选课时间,然后模拟跑一遍,每次选结束时间最早的那个即可。
\end{description}

\problem{Let the light guide us}

\begin{description}
\item[负责] 金杰、汤定一
\item[情况] 比赛中通过 - 202min(1Y)
\item[题意]
n*m的棋盘上每行要建一个塔。给每个格点的建塔费用。再给每个格点的魔法值,相邻行的两个塔的距离不能大于两点魔法值之和。求最小费用。
\item[题解]
dp[i][j]表示前i-1行都建了塔,第i行建在j上的最小总费用。每行建一棵线段树,把dp[i][j]更新到这个点左右魔法值的范围内。然后下一行每个点的费用值就是该点左右魔法值范围内的最小值+该点费用。
\end{description}

\problem{我是J题的标题}

\begin{description}
\item[负责] 负责一、负责二
\item[情况] 比赛中通过 - 233min(1Y)
\item[题意]
我是题意。
\item[题解]
我是题解。
\end{description}

\section{总结}

我是总结。

\end{document}


